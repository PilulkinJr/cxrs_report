\documentclass[../main.tex]{subfiles}

\begin{document}

\subsection{Project Structure}%
\label{sec:project_structure}

Most of the functions implemented in this project contain docstrings written in \enquote{classic} Python style using \textbf{reStructuredText} syntax. Each doscstring contains a short description of a function, list of arguments and theis types and list of exceptions which can be rised during execution of the function.

\subsection{Reading data from IMAS}%
\label{sec:reading_from_imas}

Module responsible for reading data from IMAS and creating appropriate plasma and DNB models is stored in \texttt{cxrs\//imas} directory.

\subsubsection{Supplementary Functions}%
\label{sec:imas_supp}
File \texttt{ids.py} contains only one function \texttt{ids\_get} which is used to check and load requested IDS. For example
\begin{minted}{python}
    charge_exchange_ids = ids_get(
        "charge_exchange",
        134000,
        30,
        user="public",
        database="iter",
        version="3",
        occurrence=0
    )
\end{minted}
will return an IDS object associated with charge exchange diagnostic for particular pulse from ITER public database. Each IDS can store several \texttt{occurences} (refer to the description of an IDS). For example in \texttt{charge\_exchange} IDS occurrences used to store data related to different diagnostics.

File \texttt{find\_nearest.py} contains only one function \texttt{find\_nearest}. It is used for locating nearest time slice to the time requested by user.

\subsubsection{\texttt{equilibrium} IDS}%
\label{sec:equilibrium_ids}

File \texttt{equilibrium.py} contains \texttt{EquilibriumIDS} class which is used to read data from \\ \texttt{equilibrium} IDS and create CHERAB's \href{https://cherab.github.io/documentation/plasmas/equilibrium.html?highlight=efit#cherab.tools.equilibrium.efit.EFITEquilibrium}{\texttt{EFITEquilibrium}} object via \texttt{time} method. It is later used to build core plasma model.

\begin{minted}{python}
    equilibrium_ids = EquilibriumIDS(134000, 30) 
    equilibrium = equilibrium_ids.time(-1)
\end{minted}
here \texttt{-1} for time is used to aquire a time slice in the middle of time range for this particular pulse.

\subsubsection{\texttt{core\_profiles} IDS}%
\label{sec:core_profiles_ids}

File \texttt{core\_profiles.py} contains \texttt{CoreProfilesIDS} class which is used to read data from \texttt{core\_profiles} IDS and create CHERAB's \href{https://cherab.github.io/documentation/plasmas/core_plasma_classes.html?highlight=plasma#cherab.core.Plasma}{\texttt{Plasma}} object via \texttt{create\_plasma} method. It is poses as core plasma model.

\begin{minted}{python}
    core_profiles_ids = CoreProfilesIDS(134000, 30)
    core_plasma = core_profiles_ids.create_plasma(equilibrium)
\end{minted}
\emph{Note} that \texttt{create\_plasma} method does not require time value since it uses one that stored in \texttt{equilibrium}.

One required argument of \texttt{create\_plasma} is \texttt{equilibrium}. It provides $\Psi_\text{norm}$ distribution which is used to map density, temperature and bulk velocity distributions of plasma. Functions \texttt{distribution\_density}, \texttt{distribution\_temperature} and \texttt{distribution\_velocity} are doing exactly that. \emph{Note:} \texttt{distribution\_temperature} tries to use average ion temrerature or electron temperature if species' own temperature profile is absent in the IDS.

Function \texttt{detect\_species} is used to recognise ion and neutral species from label given in IDS. Since there was no convention on the labeling at the time this appeared to be a huge problem. \texttt{detect\_species} uses regulat expressions to math species label along some other tricks. It returns CHERAB's \href{https://cherab.github.io/documentation/atomic/elements_and_isotopes.html?highlight=element#cherab.core.atomic.elements.Element}{\texttt{Element}} object and species charge as a number (0 for neutrals).

\subsubsection{\texttt{edge\_profiles} IDS}%
\label{sec:edge_profiles_ids}

File \texttt{edge\_profiles.py} contains \texttt{EdgeProfilesIDS} class which is used to read data from \texttt{edge\_profiles} IDS and create CHERAB's \href{https://cherab.github.io/documentation/plasmas/core_plasma_classes.html?highlight=plasma#cherab.core.Plasma}{\texttt{Plasma}} object via \texttt{create\_plasma} method. It is poses as edge plasma model.

\texttt{edge\_profiles} IDS requires its own class because values there set to a mesh instead of 1D profile as in \texttt{core\_profiles} IDS. Second reason is that

\begin{minted}{python}
    edge_profiles_ids = EdgeProfilesIDS(134000, 30)
    edge_plasma = edge_profiles_ids.create_plasma(time=-1)
    # or
    edge_plasma = edge_profiles_ids.create_plasma(equilibrium=equilibrium)
\end{minted}
\emph{Note} that \texttt{create\_plasma} has two optional arguments: \texttt{time} and \texttt{equilibrium}, but at least one of them is required. \texttt{time=-1} is for the same reason as in~\cref{sec:equilibrium_ids}.

\subsubsection{\texttt{charge\_exchange} IDS}%
\label{sec:charge_exchange_ids}

File \texttt{charge\_exchange.py} contains \texttt{ChargeExchangeIDS} class which is used to read data from \texttt{charge\_exchange} IDS and create different types of observers.

\begin{minted}{python}
    charge_exchange_ids = ChargeExchangeIDS(134000, 30)
    sightlines = SightlineGroup(ids=charge_exchange_ids)
\end{minted}

For information on \texttt{SightlineGroup}, see~\cref{sec:sightlines}

\subsubsection{\texttt{nbi} IDS}%
\label{sec:nbi_ids}

File \texttt{nbi.py} contains \texttt{NBIIDS} class which is used to read data from \texttt{nbi} IDS and create CHERAB's \href{https://cherab.github.io/documentation/plasmas/particle_beams.html?highlight=beam#cherab.core.Beam}{\texttt{Beam}} object via \texttt{create\_beam} method. It is poses as DNB model.

\begin{minted}{python}
    nbi_ids = NBIIDS(134000, 30)
    beam = nbi_ids.create_beam(
        time=-1,
        plasma=core_plasma,
        atomic_data=adas,
        attenuation_instructions=attenuation_instructions,
        emission_instructions=emission_instructions
    )
\end{minted}

here \texttt{time=-1} for the same reasona as in~\cref{sec:equilibrium_ids}. For more information on \texttt{adas} see~\cref{sec:atomic}, on \texttt{attenuation\_instructions} and \texttt{emission\_instructions} see~\cref{sec:emission}.

At the time \texttt{Beam} supports model with only one beamlet and \texttt{create\_beam} is designed with this in mind.

\subsection{Setting the Wall}%
\label{sec:wall}

\texttt{cxrs/machine/pfc\_mesh.py}

\subsection{Observers}%
\label{sec:observers}

\texttt{cxrs/observers}

\subsubsection{Base Class}%
\label{sec:observers_base}

\subsubsection{Sightlines}%
\label{sec:sightlines}

\subsubsection{Optics}
\subsubsection{Fibres}
\subsubsection{Camera}
\subsubsection{Others}

\subsection{Populating CHERAB's Atomic Database}%
\label{sec:atomic}

\subsection{Utility Functions}%
\label{sec:utility}

\subsubsection{Parsing XML Configuration File}%
\label{sec:xml}

\subsubsection{Setting Emission Parameters}%
\label{sec:emission}

\subsubsection{Math Functions}%
\label{sec:math}

\subsubsection{Others}%
\label{sec:other}

\end{document}