\documentclass[../../main]{subfiles}

\begin{document}

\texttt{cxrs} uses CLI with git-like structure with subprograms. They are defined in \texttt{cxrs/subprograms} module.

Script controlling CLI is \texttt{cxrs/\_\_main\_\_.py}. It defines \texttt{CLI} class and all subprograms as its methods. Each method calls \texttt{main} function from respective module and passes arguments from command line to it.

\subsubsection{\texttt{composition}}

\paragraph{\texttt{main}} function is called when CLI is used:\\
\mintinline{bash}{python -m cxrs composition -s 134000 -r 30 -f 134000_30_composition.txt}\\

\begin{minted}{python}
    main(
        shot=134000,
        run=30,
        user="public",
        database="iter",
        version="3",
        filename="134000_30_composition.txt"
    )
\end{minted}

\begin{itemize}[align=left]
    \item[\texttt{shot}] IMAS database shot ID;
    \item[\texttt{run}] IMAS database run ID;
    \item[\texttt{user}] IMAS user ID;
    \item[\texttt{database}] IMAS database ID;
    \item[\texttt{version}] IMAS major version number:
    \item[\texttt{filename}] path to output file. If not specified, output will be printed to stdout.
\end{itemize}

\emph{Note:} if \texttt{shot} and \texttt{run} are not specified program will print compositions for all available pulses.

\paragraph{\texttt{composition\_core}} function prints core plasma composition for requested pulse.

\begin{minted}{python}
    composition_core(
        shot=134000,
        run=30,
        user="public",
        database="iter",
        version="3"
    )
\end{minted}

\begin{itemize}[align=left]
    \item[\texttt{shot}] IMAS database shot ID;
    \item[\texttt{run}] IMAS database run ID;
    \item[\texttt{user}] IMAS user ID;
    \item[\texttt{database}] IMAS database ID;
    \item[\texttt{version}] IMAS major version number.
\end{itemize}

\paragraph{\texttt{composition\_edge}} function prints core plasma composition for requested pulse.

\begin{minted}{python}
    composition_edge(
        shot=134000,
        run=30,
        user="public",
        database="iter",
        version="3"
    )
\end{minted}

\begin{itemize}[align=left]
    \item[\texttt{shot}] IMAS database shot ID;
    \item[\texttt{run}] IMAS database run ID;
    \item[\texttt{user}] IMAS user ID;
    \item[\texttt{database}] IMAS database ID;
    \item[\texttt{version}] IMAS major version number.
\end{itemize}

\paragraph{\texttt{scenario\_summary}} function returns an output of \mintinline{bash}{scenario_summary -c shot,run} as a string.

\begin{minted}{python}
    scenario_summary()
\end{minted}

\subsubsection{\texttt{info}}

\paragraph{\texttt{main}} function is called when CLI is used:\\
\mintinline{bash}{python -m cxrs info -s 134000 -r 30 -c config.xml}.
It reads IMAS data for requested pulse, builds appropriate models: EFITEquilibrium, Plasma, Beam and produces useful information in form of plots and text tables. It allows to quickly look at models before performing computational-heavy observation tasks.
Functions for creating plots and tables as placed in \texttt{cxrs/utility/info} folder in respective files.

\begin{minted}{python}
    main(
        shot=134000,
        run=30,
        user="public",
        database="iter",
        version="3",
        filename="config.xml"
    )
\end{minted}

\begin{itemize}[align=left]
    \item[\texttt{shot}] IMAS database shot ID;
    \item[\texttt{run}] IMAS database run ID;
    \item[\texttt{user}] IMAS user ID;
    \item[\texttt{database}] IMAS database ID;
    \item[\texttt{version}] IMAS major version number:
    \item[\texttt{filename}] path to configuration file.
\end{itemize}

\subsubsection{\texttt{populate}}

\paragraph{\texttt{main}} function is called when CLI is used:\\
\mintinline{bash}{python -m cxrs populate}\\
It calls \texttt{populate\_more} function from \texttt{cxrs/populate/create.py} with single argument \\ \texttt{adas\_path="/work/projects/adas/adas/"}. This populates local CHERAB's atomic data repository at \texttt{/home/\$USER/.cherab} folder with data defined in the body of \texttt{populate\_more} function and makes \texttt{cxrs} use ADAS rate files stored in \texttt{/work/projects/adas/adas}.

\subsubsection{\texttt{read\_ids}}

Contain functions that are used to read necessary data from charge\_exchange and nbi IDSs. Designed for debug purposes.

\paragraph{\texttt{cxs}} function is called when CLI is used:\\
\mintinline{bash}{python -m cxrs read_ids -s 1 -r 1 -u shabasa -d test --cxs -o 0}\\
Prints all information related to diagnostic geometry stored in \texttt{charge\_exchange} IDS for requested pulse.

\begin{minted}{python}
    cxs(
        shot=1,
        run=1,
        user="shabasa",
        database="test",
        version="3",
        occurrence=0
    )
\end{minted}

\begin{itemize}[align=left]
    \item[\texttt{shot}] IMAS database shot ID;
    \item[\texttt{run}] IMAS database run ID;
    \item[\texttt{user}] IMAS user ID;
    \item[\texttt{database}] IMAS database ID;
    \item[\texttt{version}] IMAS major version number;
    \item[\texttt{occurrence}] IDS occurrence ID.
\end{itemize}

\paragraph{\texttt{nbi}} function is called when CLI is used:\\
\mintinline{bash}{python -m cxrs read_ids -s 1 -r 1 -u shabasa -d test --nbi -o 0}\\
Prints all information related to DNB geometry and parameters stored in \texttt{nbi} IDS for requested pulse.

\begin{minted}{python}
    nbi(
        shot=1,
        run=1,
        user="shabasa",
        database="test",
        version="3",
        occurrence=0
    )
\end{minted}

\begin{itemize}[align=left]
    \item[\texttt{shot}] IMAS database shot ID;
    \item[\texttt{run}] IMAS database run ID;
    \item[\texttt{user}] IMAS user ID;
    \item[\texttt{database}] IMAS database ID;
    \item[\texttt{version}] IMAS major version number;
    \item[\texttt{occurrence}] IDS occurrence ID.
\end{itemize}

\subsubsection{\texttt{write\_ids}}

Contain functions that are used to write \texttt{charge\_exchange} and \texttt{nbi} data from configuration file to appropriate IDSs.

\paragraph{\texttt{cxs}} function is called when CLI is used:\\
\mintinline{bash}{python -m cxrs write_ids -c config.xml -s 1 -r 1 -u shabasa -d test --cxs -o 0}\\
Creates a \texttt{charge\_exchange} IDS instance in a local database with contents of the configuration file.

\begin{minted}{python}
    cxs(
        shot=1,
        run=1,
        user="shabasa",
        database="test",
        version="3",
        config="config.xml",
        occurrence=0
    )
\end{minted}

\begin{itemize}[align=left]
    \item[\texttt{shot}] IMAS database shot ID;
    \item[\texttt{run}] IMAS database run ID;
    \item[\texttt{user}] IMAS user ID;
    \item[\texttt{database}] IMAS database ID;
    \item[\texttt{version}] IMAS major version number;
    \item[\texttt{config}] path to configuration file;
    \item[\texttt{occurrence}] IDS occurrence ID.
\end{itemize}

\paragraph{\texttt{nbi}} function is called when CLI is used:\\
\mintinline{bash}{python -m cxrs write_ids -c config.xml -s 1 -r 1 -u shabasa -d test --nbi -o 0}\\
Creates a \texttt{nbi} IDS instance in a local database with contents of the configuration file.

\begin{minted}{python}
    nbi(
        shot=1,
        run=1,
        user="shabasa",
        database="test",
        version="3",
        config="config.xml",
        occurrence=0
    )
\end{minted}

\begin{itemize}[align=left]
    \item[\texttt{shot}] IMAS database shot ID;
    \item[\texttt{run}] IMAS database run ID;
    \item[\texttt{user}] IMAS user ID;
    \item[\texttt{database}] IMAS database ID;
    \item[\texttt{version}] IMAS major version number;
    \item[\texttt{config}] path to configuration file;
    \item[\texttt{occurrence}] IDS occurrence ID.
\end{itemize}
\subsubsection{\texttt{search}}

Search for pulses that contain requested element by its symbol.

\paragraph{\texttt{main}} function is called when CLI is used:\\
\mintinline{bash}{python -m cxrs search C}\\
Prints list of pulses with their shot and run numbers. Uses \mintinline{bash}{scenario_summary} program.

\begin{minted}{python}
    main(
        symbol="C",
        verbose=False,
        clean=False,
    )
\end{minted}

\begin{itemize}[align=left]
    \item[\texttt{symbol}] symbol of the requested element;
    \item[\texttt{verbose}] print additional text;
    \item[\texttt{clean}] delete temporary file produced by \mintinline{bash}{scenario_summary} program.
\end{itemize}

\subsubsection{\texttt{simulate}}

\paragraph{\texttt{main}} function is called when CLI is used:\\
\mintinline{bash}{python -m cxrs simulate -s 134000 -r 30 -t -1 -c config.xml}\\
Main routine for CXRS spectra simulation. Results are stored in \texttt{cxrs\_output/shot\_<shot>\_run\_<run>/time\_<time>} directory.

\begin{minted}{python}
    main(
        shot=134000,
        run=30,
        time=-1.0,
        user=134000,
        database=134000,
        version=134000,
        config="config.xml",
    )
\end{minted}

\begin{itemize}[align=left]
    \item[\texttt{shot}] IMAS database shot ID;
    \item[\texttt{run}] IMAS database run ID;
    \item[\texttt{time}] requested time;
    \item[\texttt{user}] IMAS user ID;
    \item[\texttt{database}] IMAS database ID;
    \item[\texttt{version}] IMAS major version number;
    \item[\texttt{config}] path to configuration file.
\end{itemize}
\emph{Note:} \mintinline{bash}{-t -1} can be used to choose time slice in the middle of the time range or it case when IDS contains only one time slice.

\paragraph{\texttt{print\_parameters}} function is used to print simulation parameters defined in the configuration file to the stdout.

\begin{minted}{python}
    main(
        user_options,
        plasma_parameters,
        beam_parameters,
        emission_parameters
    )
\end{minted}

\begin{itemize}[align=left]
    \item[\texttt{user\_options}] parameters from <user\_options> section of the configuration file;
    \item[\texttt{plasma\_parameters}] parameters from <plasma\_paragrameters> section of the configuration file;
    \item[\texttt{beam\_parameters}] parameters from <beam\_parameters> section of the configuration file;
    \item[\texttt{emission\_parameters}] parameters from <emission\_parameters> section of the configuration file.
\end{itemize}

\end{document}
