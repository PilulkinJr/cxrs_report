\documentclass[../../main.tex]{subfiles}

\begin{document}
\subsubsection{Getting Information on Plasma Parameters}%
\label{sec:info}

This module contains routines used by \mintinline{bash}{cxrs info} subprogram.

\dirtree{%
    .1 info/.
    .2 \_\_init\_\_.py.
    .2 beam.py.
    .2 equilibrium.py.
    .2 passive.py.
    .2 plasma.py.
    .2 profiles.py.
}

\paragraph{\texttt{beam.py}} file contains \texttt{beam\_info} function designed to produce DNB related information for a model built for a simulation.

\begin{minted}{python}
    beam_info(
        plasma=core_plasma,
        id_dict = {
            "shot": 134000,
            "run": 30,
            "user": "public",
            "database": "iter",
            "version": "3",
        },
        plot_dict={
            "xlims": (4.0, 8.5),
            "ylims": (-4.5, 4.5),
            "resolution": 0.01,
            "beam_slice": 0.5,
        },
        dirname="beam_info",
        config="config.xml",
    )
\end{minted}

\begin{itemize}[align=left]
    \item[\texttt{plasma}] cherab's Plasma object;
    \item[\texttt{id\_dict}] dictionary with IMAS IDs;
    \item[\texttt{plot\_dict}] dictionary with plot parameters;
    \item[\texttt{dirname}] path to save directory;
    \item[\texttt{config}] path to configuration file.
\end{itemize}

\paragraph{\texttt{equilibrium.py}} file contains \texttt{equilibrium\_info} function designed to produce equilibrium related information for a model built for a simulation.

\begin{minted}{python}
    equilibrium_info(
        id_dict = {
            "shot": 134000,
            "run": 30,
            "user": "public",
            "database": "iter",
            "version": "3",
        },
        plot_dict={
            "xlims": (4.0, 8.5),
            "ylims": (-4.5, 4.5),
            "resolution": 0.01,
            "beam_slice": 0.5,
        },
        dirname="equilibrium_info",
        filename=None,
    )
\end{minted}

\begin{itemize}[align=left]
    \item[\texttt{id\_dict}] dictionary with IMAS IDs;
    \item[\texttt{plot\_dict}] dictionary with plot parameters;
    \item[\texttt{dirname}] path to save directory;
    \item[\texttt{filename}] name of the produced file.
\end{itemize}

\paragraph{\texttt{passive.py}} file contains \texttt{passive\_info} function designed to produce passive charge-exchange related information for a model built for a simulation.

\begin{minted}{python}
passive_info(
    plasma=core_plasma,
    beam=beam,
    id_dict = {
        "shot": 134000,
        "run": 30,
        "user": "public",
        "database": "iter",
        "version": "3",
    },
    plot_dict={
        "xlims": (4.0, 8.5),
        "ylims": (-4.5, 4.5),
        "resolution": 0.01,
        "beam_slice": 0.5,
    },
    dirname="passive_info",
)
\end{minted}

\begin{itemize}[align=left]
    \item[\texttt{plasma}] CHERAB's \texttt{Plasma} object;
    \item[\texttt{beam}] CHERAB's \texttt{Beam} object;
    \item[\texttt{id\_dict}] dictionary with IMAS IDs;
    \item[\texttt{plot\_dict}] dictionary with plot parameters;
    \item[\texttt{dirname}] path to save directory;
\end{itemize}

\paragraph{\texttt{profiles.py}} file contains \texttt{profiles\_info} function designed to produce plasma profiles.

\begin{minted}{python}
profiles_info(
    core_plasma=core_plasma,
    edge_plasma=edge_plasma,
    beam=beam,
    id_dict = {
        "shot": 134000,
        "run": 30,
        "user": "public",
        "database": "iter",
        "version": "3",
    },
    plot_dict={
        "xlims": (4.0, 8.5),
        "ylims": (-4.5, 4.5),
        "resolution": 0.01,
        "beam_slice": 0.5,
    },
    dirname="profiles",
)
\end{minted}

\begin{itemize}[align=left]
    \item[\texttt{core\_plasma}] CHERAB's \texttt{Plasma} object for core plasma;
    \item[\texttt{edge\_plasma}] CHERAB's \texttt{Plasma} object for edge plasma;
    \item[\texttt{beam}] CHERAB's \texttt{Beam} object;
    \item[\texttt{id\_dict}] dictionary with IMAS IDs;
    \item[\texttt{plot\_dict}] dictionary with plot parameters;
    \item[\texttt{dirname}] path to save directory;
\end{itemize}

\subsubsection{Parsing XML Configuration File}%
\label{sec:xml}

File \texttt{utility/xml.py} contains functions to work with configuration files. For XML parsing module \texttt{xml.etree} from Python's standard library is used.

\mint{python}{srt2bool(string)}
converts "on" to \mintinline{python}{True} and "off" to \mintinline{python}{False}.

\mint{python}{bool2str(arg)}
converts \mintinline{python}{True} to "on" and \mintinline{python}{False} to "off".

\mint{python}{read_xml_entry(tree_element)}
reads XML structure and returns Python object based on structure's description.

\mint{python}{parse_user_options(section, config)}
reads \texttt{<user\_options>} \texttt{section} from the configuration file and returns dictionary with its parameters.

\mint{python}{parse_emission_lines(config)}
reads \texttt{<emission\_lines>} section of the configuration file and returns list of CHERAB's \href{https://cherab.github.io/documentation/atomic/emission_lines.html?highlight=line#cherab.core.atomic.line.Line}{\texttt{Line}} objects.

\mint{python}{parse_diagnostic_geometry(config)}
reads \texttt{<geometry>} \texttt{section} from the configuration file and returns dictionary with its parameters.

\mint{python}{parse_dnb_parameters(config)}
reads \texttt{<DNB>} \texttt{section} from the configuration file and returns dictionary with its parameters.

\mint{python}{parse_wavelength_ranges(config)}
reads \texttt{<wavelength\_ranges>} \texttt{section} from the configuration file and returns dictionary with its parameters.


\subsubsection{Setting Emission Parameters}%
\label{sec:emission}

\begin{minted}{python}
    plasma_emission_parameters = plasma_emission(
        plasma,
        lines,
        bremsstrahlung=True,
        recombination=True,
        excitation=True,
        passive=True,
    )
\end{minted}

\begin{itemize}[align=left]
    \item[\texttt{plasma}] cherab's Plasma object;
    \item[\texttt{lines}] list of cherab's Line objects;
    \item[\texttt{bremsstrahlung}] turn Bremsstrahlung on;
    \item[\texttt{recombination}] turn recombination emission on;
    \item[\texttt{excitation}] turn excitation emission on;
    \item[\texttt{passive}] turn passive emission on.
\end{itemize}

\begin{minted}{python}
beam_emission_parameters = beam_emission(
    plasma,
    lines,
)
\end{minted}

\begin{itemize}[align=left]
    \item[\texttt{plasma}] cherab's Plasma object;
    \item[\texttt{lines}] list of cherab's Line objects;
\end{itemize}

\subsubsection{Math Functions}%
\label{sec:math}

\mint{python}{to_cylindrical(point)}
convert Raysect's Point3D from cartesian coordinates to cylindrical.

\mint{python}{to_cylindrical_multiple(points)}
convert list of Raysect's Point3D from cartesian coordinates to cylindrical.

\mint{python}{to_cartesian(point)}
convert Raysect's Point3D from cylindrical coordinates to cartesian.

\mint{python}{to_cartesian_multiple(points)}
convert list of Raysect's Point3D from cylindrical coordinates to cartesian.

\begin{minted}{python}
ion_temperature(
    doppler_width,
    natural_wavelength,
    atomic_weight
)
\end{minted}

Calculate impurity's ion temperature by Doppler width of observed spectral line: \\
$T_\text{i} = m c^2 \left( \frac{\Delta \lambda}{\lambda_0} \right)^2$

\begin{itemize}[align=left]
    \item[\texttt{doppler\_width}] width of line's Doppler broadening in nm;
    \item[\texttt{natural\_wavelength}] line's \enquote{natural} wavelength in nm;
    \item[\texttt{atomic\_weight}] impurity's atomic weight.
\end{itemize}

\begin{minted}{python}
velocity_edge(
    natural_wavelength,
    doppler_shift_upper,
    doppler_shift_lower,
    angle_upper,
    angle_lower,
)
\end{minted}

Calculate plasma bulk velocity at the edge:
\begin{align*}
    v_{\text{tor}} & = \frac{c}{\lambda_0} \frac{\Delta \lambda_{\text{upper}} \sin(\alpha_{\text{lower}}) + \Delta \lambda_{\text{lower}} \sin(\alpha_{\text{upper}})}{\sin(\alpha_{\text{upper}} + \alpha_{\text{lower}})} \\
    v_{\text{pol}} & = \frac{c}{\lambda_0} \frac{\Delta \lambda_{\text{upper}} \cos(\alpha_{\text{lower}}) - \Delta \lambda_{\text{lower}} \cos(\alpha_{\text{upper}})}{\sin(\alpha_{\text{upper}} + \alpha_{\text{lower}})}
\end{align*}

\begin{itemize}[align=left]
    \item[\texttt{natural\_wavelength}] line's \enquote{natural} wavelength in nm;
    \item[\texttt{doppler\_shift\_upper}] Doppler shift in nm registered by upper diagnostic;
    \item[\texttt{doppler\_shift\_lower}] Doppler shift in nm registered by lower diagnostic;
    \item[\texttt{angle\_upper}] angle in rad between upper diagnostic's line of sight and toroidal direction;
    \item[\texttt{angle\_lower}] angle in rad between lower diagnostic's line of sight and toroidal direction;
\end{itemize}

\begin{minted}{python}
velocity_core(
    doppler_shift
    angle,
    natural_wavelength,
)
\end{minted}

Calculate plasma bulk velocity at the edge:
$v = \frac{c}{\cos(\alpha)} \frac{\Delta \lambda}{\lambda_0}$

\begin{itemize}[align=left]
    \item[\texttt{doppler\_shift}] Doppler shift in nm registered by diagnostic;
    \item[\texttt{angle}] angle in rad between diagnostic's line of sight and toroidal direction;
    \item[\texttt{natural\_wavelength}] line's \enquote{natural} wavelength in nm.
\end{itemize}

\begin{minted}{python}
lines_intersection(
    start_a,
    vector_a,
    start_b,
    vector_b,
    cylindrical=False,
)
\end{minted}

Find point of intersection between two lines. Lines are defined by starting point and direction vector: $\vec{r} = \vec{r_0} + \vec{a}t$.

\begin{itemize}[align=left]
    \item[\texttt{start\_a}] start point of the first line;
    \item[\texttt{vector\_a}] direction vector of the first line;
    \item[\texttt{start\_b}] start point of the second line;
    \item[\texttt{vector\_b}] direction vector of the second line;
    \item[\texttt{cylindrical}] return intersection point in cylindrical coordinates.
\end{itemize}

\begin{minted}{python}
line_cylinder_intersection(
    start, 
    vector, 
    radius, 
    cylindrical=False
)
\end{minted}

Find point of intersection between a line and a cylinder. Line is defined by starting point and direction vector: $\vec{r} = \vec{r_0} + \vec{a}t$. Cylinder's axis coincides with Z axis.

\begin{itemize}[align=left]
    \item[\texttt{start}] start point of the line;
    \item[\texttt{vector}] direction vector of the line;
    \item[\texttt{radius}] radius of the cylinder;
    \item[\texttt{cylindrical}] return intersection point in cylindrical coordinates.
\end{itemize}

\begin{minted}{python}
    tangency_to_cartesian(
        point, 
        tangency_radius, 
        angle, 
        direction
    )
\end{minted}

Convert NBI IDS beam direction parameters to vector in cartesian coordinates.

\begin{itemize}[align=left]
    \item[\texttt{point}] Beam source point;
    \item[\texttt{tangency\_radius}] Tangency radius (major radius where the central line of a NBI unit is tangent to a circle around the torus) [m];
    \item[\texttt{angle}] Angle of inclination between a beamlet at the centre of the injection unit surface and the horizontal plane [rad];
    \item[\texttt{direction}] Direction of the beam seen from above the torus: -1 = clockwise; 1 = counter clockwise.
\end{itemize}

\begin{minted}{python}
    cartesian_to_tangency(point, vector)
\end{minted}

Convert beam direction vector in cartesian coordinates to NBI IDS parameters.

\begin{itemize}[align=left]
    \item[\texttt{point}] Beam source coordinates point [m];
    \item[\texttt{vector}] Beam direction vector [m].
\end{itemize}

\begin{minted}{python}
    convolve(data, kernel)
\end{minted}

Convolve without edge effects.

\begin{itemize}[align=left]
    \item[\texttt{data}] Data array;
    \item[\texttt{kernel}] Kernel array.
\end{itemize}

\subsubsection{Fitting Routine}%
\label{sec:fitting}

\begin{minted}{python}
    fit(
        lines,
        observer_groups,
        atomic_data,
        background=True,
        fit_report=False,
        dirname=None,
        scenario=None,
        save=True,
    )
\end{minted}

Fit emission spectra and reconstruct plasma parameters.

\begin{itemize}[align=left]
    \item[\texttt{lines}] List of beam emission lines.
    \item[\texttt{observer\_groups}] List of observer groups.
    \item[\texttt{atomic\_data}] Atomic data object.
    \item[\texttt{fit\_report}] If True, plot fit report. (Default value = False)
    \item[\texttt{dirname}] Path to save directory. (Default value = None)
    \item[\texttt{scenario}] Dictionary with scenario parameters. (Default value = None)
    \item[\texttt{save}] If True, save results in image format. (Default value = True)
\end{itemize}

\mint{python}{find_nearest(array, value)}
Find index of the \texttt{array} element with value closest to \texttt{value}.

\mint{python}{background_model(wavelengths, spectrum)}
Define model for background (Bremsstrahlung) fitting: $B(\lambda) = A / \lambda$.

\begin{minted}{python}
    reference_line_model(
        line_label,
        line_data,
        wavelengths,
        spectrum,
        bg_model=None
    )
\end{minted}

\begin{minted}{python}
    def line_model(
        line_label,
        line_data,
        reference_line_label,
        reference_line_data,
        wavelengths,
        spectrum,
        bg_model=None,
    )
\end{minted}

\subsubsection{Others}%
\label{sec:other}

utility/annotation
utility/data
utility/timer
utility/fit
\end{document}