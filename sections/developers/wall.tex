\documentclass[../../main]{subfiles}

\begin{document}
File \texttt{cxrs/machine/pfc\_mesh.py} contains function \texttt{load\_pfc\_mesh} that is used to create a reactor wall model. All meshes are stored in \texttt{machine/portplugs} and \texttt{machine/simple} subfolders.

\begin{minted}{python}
    wall = load_pfc_mesh(
        path=os.path.join(os.path.dirname(__file__)),
        reflections=True,
        roughness={"Be": 0.26, "W": 0.29, "Ss": 0.13},
        detailed=False,
        parent=world,
        transform=None,
        name="Wall",
    )
\end{minted}

\begin{itemize}[align=left]
    \item[\texttt{path}] is a path to .rsm mesh files;
    \item[\texttt{reflections}] sets reflections on or off;
    \item[\texttt{roughness}] is a roughness dictionary for PFC materials ("Be", "W", "Ss");
    \item[\texttt{detailed}] -- Load the detailed wall model instead of the simple one;
    \item[\texttt{parent}] is a parent node;
    \item[\texttt{transform}] is a transformation matrix;
    \item[\texttt{name}] is a name of this plasma.
\end{itemize}
Here \texttt{reflections} assigns material properties to appropriate wall segments if set to \texttt{True} and sets the wall as \href{https://raysect.github.io/documentation/api_reference/optical/optical_surfaces.html?highlight=absorber#raysect.optical.material.absorber.AbsorbingSurface}{perfect absorber} if set to \texttt{False}. It is used to effectively turn reflections on and off. \texttt{roughness} argument sets \href{https://raysect.github.io/documentation/demonstrations/materials/surface_roughness.html?highlight=rough}{\emph{roughness}} of the materials assigned to the wall segments (beryllium, tungsten and stainless steel).
For more information on \texttt{parent}, \texttt{transform} and \texttt{name} refer to Raysect's documentation on \href{https://raysect.github.io/documentation/api_reference/edge/raysect_edge_scenegraph.html?highlight=node#raysect.edge.scenegraph.node.Node}{\texttt{Node}}.
\end{document}
