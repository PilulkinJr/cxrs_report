\documentclass[../main.tex]{subfiles}

\begin{document}

\subsection{CXRS Diagnostics}
\subsubsection{CXRS for Plasma Measurements}
The Charge Exchange Recombination Spectroscopy (CXRS) diagnostic measures
line emissions of several low $Z$ impurities in the plasma due to interaction
with a neutral beam.

Measured parameters:
\begin{itemize}
    \item Line’s width~--~ion temperature;
    \item Line center’s shift~–-~toroidal and poloidal velocities;
    \item Line’s intensity~–-~impurity's density and $Z_\text{eff}$.
\end{itemize}

\begin{table}[ht]
    \centering
    % \caption{}%
    % \label{}
    % \begin{tabularx}{0.5\textwidth}{l c r} \toprule
    \begin{tabular}[]{l l l}
        \toprule
        % Ion         & Transition & Wavelength, nm \\
        % \midrule
        BeIV        & $6\to5$   & 465.8 nm \\
        BeIV        & $8\to6$   & 468.5 nm \\
        HeII        & $4\to3$   & 468.5 nm \\
        \midrule
        ArXVIII     & $16\to15$ & 522.5 nm \\
        NeX         & $11\to10$ & 524.9 nm \\
        CVI         & $8\to7$   & 529.1 nm \\
        \midrule
        H$\upalpha$ &           & 656.3 nm \\
        MSE         &           & 659.1 nm \\
        \bottomrule
    \end{tabular}
    % \end{tabularx}
\end{table}

\subsubsection{CXRS in ITER}
\paragraph{Geometry}
\paragraph{Equipment}
\paragraph{IMAS Database}

\subsection{Development of the New Simulation Code}
\subsubsection{Existing Code}
Simulation of Spectra (SOS) code by M. G. von Hellermann~\cite{sos}

Features:
\begin{itemize}
    \item Simulation takes into account many physical effects (halo effect, crossection effect, plume effect and others);
    \item Written in Matlab;
    \item Has Graphical User Interface~(\cref{fig:sos_interface}).
\end{itemize}

\begin{figure}[ht]
    \centering
    \includegraphics[width=0.75\textwidth]{images/sos_interface}
    \caption{SOS interface.}%
    \label{fig:sos_interface}
\end{figure}

\subsubsection{Motivation}
Existing code (Simulation of Spectra - SOS) lacks some features:
\begin{itemize}
    \item Simplified plasma, tokamak and diagnostic geometry \\
          (e.g. elliptical plasma, point emission and others);
    \item Does not take reflections into account;
    \item Cannot use data from IMAS directly;
    \item Requires Matlab license, hard to extend by new developers.
\end{itemize}

The goal was to create an open and extensible simulation code using Python.

Sub goals:
\begin{itemize}
    \item Implement interaction with IMAS database (read and write);
    \item Use IMAS data to create a plasma and diagnostic beam with spatial distributions;
    \item Use a ray-tracing engine to simulate spectra, this includes how reflections affect simulated spectra;
    \item Ensure that emission models include all physics already captured by SOS.
\end{itemize}


\subsection{Raysect and CHERAB}

\textbf{Raysect}~\cite{raysect} is a ray-tracing framework for Python designed for scientific purposes.
\begin{itemize}
    \item Supports scientific ray-tracing of spectra from physical light sources such as plasmas.
    \item Easily extensible, written with user customisation of materials and emissive sources in mind.
    \item Different observer types supported such as pinhole cameras and optical fibres.
\end{itemize}

\textbf{CHERAB}~\cite{cherab} is a Python library for forward modelling diagnostics based on spectroscopic plasma emission which provides physical models for Raysect.
Provided models for Raysect:
\begin{itemize}
    \item Tools for plasma and diagnostic beam simulations;
    \item Physical emission models (active charge exchange, bremsstrahlung and more).
\end{itemize}


\begin{figure}
    \centering
    \includegraphics[width=0.75\textwidth]{images/raysect_demo}
    \caption{Demonstration of Raysect features.}%
    \label{fig:raysect_demo}
\end{figure}

\end{document}