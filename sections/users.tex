\documentclass[../main.tex]{subfiles}

\begin{document}

\subsection{Installation}

Clone this repository then go to its root folder:

\begin{minted}{bash}
    git clone ssh://git@git.iter.org/diag/cxrs.git
    cd cxrs
\end{minted}

First of all you have to load all required modules:

\begin{minted}{bash}
    source env.sh
\end{minted}

This will purge all loaded modules and load ones that necessary to install and use \texttt{cxrs}.

To install \texttt{cxrs} on your computer run

\begin{minted}{bash}
    pip install --user .
\end{minted}

while in the the \texttt{cxrs} root directory.

\emph{Note:} you can omit \texttt{python -m} part later by adding \texttt{\$HOME/.local/bin} to your \texttt{PATH}. To do so, locate file \texttt{.bashrc} in your home directory and add the following line at the end of the file:

\begin{minted}{bash}
    export PATH="$PATH:/home/ITER/<username>/.local/bin"
\end{minted}

where \texttt{<username>} is your username at ITER GPC. You can look at it by executing next command via command line: \texttt{echo \$USER}.

\subsection{Preparations for the first run}

\begin{enumerate}[label = Step \arabic*:, align = left]

    \item Create environment file

          In order to create environment file, run

          \begin{minted}{bash}
            python -m cxrs create-env
          \end{minted}

          This will create the default environment file, that loads all needed modules. To do so, run

          \begin{minted}{bash}
            source env.sh
          \end{minted}

          It will purge all loaded modules and will load modules necessary to use \texttt{cxrs}.

          \emph{Note:} you have to do this for every new session on ITER GPC.


    \item {Create configuration file}

          \texttt{cxrs} behavior controlled by configuration file which is necessary to run the code.

          To create default configuration file, run

          \begin{minted}{bash}
            python -m cxrs create-config
          \end{minted}

          This will create a configuration file with a name \texttt{config.xml} in current directory.


    \item {Populate local atomic database}

          \texttt{cxrs} uses a local atomic database. To create one, run

          \begin{minted}{bash}
            python -m cxrs populate
          \end{minted}

          Atomic data will be copied to the \texttt{\$HOME/.cherab} directory.

\end{enumerate}

\subsection{Usage}

\subsubsection{Performing a simulation}

Main part of \texttt{cxrs} is simulation routine, to use it for pulse with \texttt{<shot>} shot number and \texttt{<run>} run number for time slice \texttt{<time>} use:

\begin{minted}{bash}
    python -m cxrs simulate -s <shot> -r <run> -t <time> -c <config>
\end{minted}

where \texttt{<config>} is a path to configuration file.

It will perform a simulation and will store result in newly created folder \texttt{cxrs\_output}.

\emph{Note:} \texttt{time} can be omitted. In that case time slice in the middle of the time range will be used. This is particularly useful for time slices with a lot of digits after after decimal separator. For example :
\begin{minted}{bash}
    python -m cxrs simulate -s 134000 -r 30 -c config.xml
\end{minted}

For additional information use help:
\begin{minted}{bash}
    python -m cxrs simulate --help
\end{minted}

\subsubsection{Print plasma profiles}

\texttt{cxrs info} subprogram can be used to inspect different distributions of plasma and DNB parameters. To use it, run

\begin{minted}{bash}
    python -m cxrs info -s <shot> -r <run>
\end{minted}

It will produce variety of plots and tables and place it in \texttt{cxrs\_output} folder.
\emph{Note:} 1D profiles represent values at the DNB axis.

\subsubsection{Print plasma composition}

You can quickly look at plasma composition for requested pulse by using

\begin{minted}{bash}
    python -m cxrs composition -s <shot> -r <run>
\end{minted}

Omit arguments to print composition for all pulses available by \texttt{scenario\_summary} program.

\subsubsection{Configuration}

\texttt{cxrs} accepts user's configurations as xml-file.

To change option's value, change value attribute in double quotes corresponding to the option.

For more information on the meaning of certain parameter, look at its description.

\emph{Note:} type attribute is actually important. For string values use \texttt{str}, for floating-point values use \texttt{float} and \texttt{int} for integers.

\emph{Warning:} Do not change tags, like \texttt{<user\_options>} or others, this will stop the program from working.

\subsubsection{Emission parameters}

Section \texttt{emission\_parameters} controls which emission types are used during a simulation. Each of them can be turned on or off separately.

\subsubsection{Plasma, beam, fibre, optics, camera, scanner and spectrometer parameters}

Each of those sections controls appropriate aspect of simulation. For information on each parameter look at its description in configuration file.

\subsubsection{Emission lines}

List of emission lines suited for observations is placed in \texttt{<emission\_parameters>} section:

\begin{minted}{xml}
<emission_lines description="CXRS elements to observe.">
    <HI>
        <name
            description="Name of the element."
            unit="n.a."
            type="str"
            value="hydrogen"/>
        <charge
            description="Charge of the ion."
            unit="n.a."
            type="int"
            value="0"/>
        <transition_levels
            description="Transition levels [upper, lower]."
            unit="n.a."
            type="int"
            value="[3, 2]"/>
    </HI>
    ... more lines omitted
</emission_lines>
\end{minted}


You can change and add new entries in this section following given template. Added emission lines will be simulated if atomic data exists for them.

\subsubsection{DNB}

DNB section of the configuration file contains parameters for Diagnostic Neutral Beam. Structure of this section replicates structure of nbi IDS.

\subsubsection{Wavelength ranges and geometry}

These two sections contain all information for CXRS diagnostics.

\end{document}
